\section{Testing Methodology}\label{sec:testing}

\subsection{Test Environment}\label{subsec:test-environment}

The testing environment was containerized using Docker to ensure reproducibility across different systems. The container configuration included:

\begin{itemize}
    \item \textbf{Base Image}: Ubuntu 22.04 LTS
    \item \textbf{Hadoop Version}: 3.3.6 (pseudo-distributed mode)
    \item \textbf{Java}: OpenJDK 11
    \item \textbf{Python}: 3.10 with required dependencies
\end{itemize}

\begin{lstlisting}[style=bashstyle,caption={Docker Compose Configuration},label={lst:docker-compose}]
version: '3'
services:
  hadoop-test:
    image: hadoop-base
    build:
      context: .
      dockerfile: Dockerfile
    ports:
      - "9870:9870"  # NameNode
      - "9864:9864"  # DataNode
      - "8088:8088"  # ResourceManager
    volumes:
      - ./data:/data
      - ./output:/output
\end{lstlisting}

\subsection{Test Cases}\label{subsec:test-cases}

\subsubsection{Functional Validation}
\begin{table}[H]
    \centering
    \caption{Functional Test Cases}
    \begin{tabular}{llll}
        \toprule
        \textbf{ID} & \textbf{Description} & \textbf{Input} & \textbf{Expected Result} \\
        \midrule
        TC-01 & Single document processing & 1 KB text file & Correct word-file mappings \\
        TC-02 & Multi-document processing & 10 files (1 MB total) & Combined index with no duplicates \\
        TC-03 & Special character handling & File with punctuation & Properly normalized terms \\
        TC-04 & Empty file handling & 0 KB file & Skipped with warning \\
        TC-05 & Large file stress test & 1 GB Wikipedia dump & Successful completion \\
        \bottomrule
    \end{tabular}
    \label{tab:functional-tests}
\end{table}

\subsubsection{Performance Benchmarking}
Tests were executed with varying dataset sizes:

\begin{lstlisting}[style=bashstyle,caption={Test Execution Command},label={lst:inverted-test}]
# Hadoop Version
time hadoop jar InvertedIndex.jar /input /output

# Python Version
python inverted_index.py --input /data --output /results
\end{lstlisting}

Execution Time Comparison (Hadoop vs Python)

\subsection{Validation Methodology}\label{subsec:validation-methodology}

\paragraph{Docker-based Testing Pipeline:}
\begin{enumerate}
    \item Container initialization with test datasets mounted
    \item Automated test execution via Makefile:

    \begin{lstlisting}[style=makefilestyle,caption={Makefile Test Targets},label={lst:docker-run}]
test-hadoop:
    docker-compose run hadoop-test \
        hadoop jar /code/InvertedIndex.jar /test-input /test-output

test-python:
    docker-compose run hadoop-test \
        python /code/inverted_index.py --input /test-input
    \end{lstlisting}

    \item Output verification using automated scripts:

    \begin{lstlisting}[language=python,caption={Verification Script},label={lst:python-test}]
def verify_index(index):
    assert 'search' in index, "Common term missing"
    assert len(index['the']) > 0, "Stopword not processed"
    \end{lstlisting}
\end{enumerate}

\subsection{Results}\label{subsec:results}

The Docker environment successfully validated:

\begin{itemize}
    \item Consistent execution across 5 test runs (σ < 2\% variation)
    \item Correct index generation for all test cases
    \item Hadoop outperformed Python by 3.7x on 1GB dataset
    \item Resource usage remained within container limits (8GB RAM allocated)
\end{itemize}

\begin{table}[H]
    \centering
    \caption{Resource Utilization}
    \begin{tabular}{lrrr}
        \toprule
        \textbf{Dataset} & \textbf{CPU Usage (\%)} & \textbf{RAM (GB)} & \textbf{Time (s)} \\
        \midrule
        1 MB & 12 & 1.2 & 4.7 \\
        100 MB & 68 & 3.8 & 28.1 \\
        1 GB & 92 & 7.5 & 193.4 \\
        \bottomrule
    \end{tabular}
    \label{tab:resources}
\end{table}

